\documentclass{article}
\usepackage[
backend=biber,
style=alphabetic,
sorting=ynt
]{biblatex}
\usepackage{indentfirst}



\addbibresource{bibliography.bib}

\title{Bibliography Summary}
\author{Ngan Nguyen}
\date{January 2022}

\begin{document}


\maketitle

\section*{Summary}
% First summary: Evaluation of mobile app paradigms 
In the research paper Evaluation of mobile app paradigms \cite{10.1145/2428955.2428968}, the authors main purpose is to help the audiences understand the four main mobile application paradigms which is mobile native applications, mobile widgets, mobile web applications, and HTML5 mobile applications. "A mobile native application or native app is an application
specifically developed to execute on a specific device platform" \cite{10.1145/2428955.2428968}. Mobile widget is a quicker way to access new, podcast, or weather content. Mobile web application is used to deliver information and services. The World Wide Web Consortium (W3C) has defined HTML5 as a set of capabilities that can do all of the jobs that existing technologies can handle in mobile Web apps. The authors evaluation: for developer, the easiest application to develop is widget; for user, the easiest application to use is native apps; for the service provider, HTML5 have the advantage of distribution. In conclusion, the authors believe that "HTML5 mobile apps keep their first places in the race of mobile paradigms" \cite{10.1145/2428955.2428968}. The authors also predict that mobile web app will soon become history and replaced by HTML5 mobile app. 

%used to skip line 
\medskip

% Second summary: Software engineering issues for mobile application development
The paper "Software engineering issues for mobile application development" \cite{10.1145/1882362.1882443} presents an overview of key software engineering research challenges relating to the development of mobile app development. The authors pointed out some additional requirement that mobile applications need such as potentially interact with other apps, sensing handling, native and hybrid, security, complexity of testing, and power consumption. The paper suggests the best practice for mobile application is agile approach. The authors think that there are a lot of space for researching in mobile development area namely user-experience , non-functional requirements, as well as process, tool, and architecture. "While the large number of mobile applications makes it appear
that software development processes for them are well understood, there remain a large number of complex issues where further work is needed."\cite{10.1145/1882362.1882443}, the author concluded.

%used to skip line 
\medskip

In the paper "What can Android mobile app developers do about the energy consumption of machine learning?", the author introduce some API and library that used to implement machine learning algorithm with mobile app. One example of the framework that can use machine learning algorithm with mobile app is Neural Networks API. Apple has released CoreML (Inc. 2017) for iOS \cite{mcintosh2019can}. The author also introduce apps that employ machine learning like Google Play Crawler and App Brain. One of the main concern of using such an enormous amount of data is the energy that would cost for the mobile device. The author concluded that Naïve Bayes and J48 are the best algorithm implementations tested to use for applications trying to reduce energy use \cite{mcintosh2019can}. If we're focus more on accuracy, the author suggested MLP had the highest average accuracy overall, with an average classification accuracy of over 95\% and an average kappa of over 0.92. \cite{mcintosh2019can}. The researcher concluded that different learning algorithm do different thing, it is up to developers choose how to use it.


%used to skip line 
\medskip

In the article "The mobile apps industry: A case study", the author discussed about the evolution, the basic, the impact of the mobile app industry. The different between Android app and iPhone app \cite{rakestraw2013mobile}. They also discussed the market place of big tech like Facebook, Amazon, Niche and the consumer reference. Revenue that these app generated. Network provider play an important role in order for these big tech to be successful. The author also brought up security and privacy, the primary user concerned. They also did some research about trends and predict the future. Based on the research, mobile app consumption has beaten web application consumption by 8\%. 

\medskip

Many smartphone apps collect potentially sensitive personal data and send it to cloud servers. However, most mobile users have a poor understanding of why their data is being collected. We present MobiPurpose, a novel technique that can take a network request made by an Android app and then classify the data collection purposes, as one step towards making it possible to explain to non-experts the data disclosure contexts. Our purpose inference works by leveraging two observations: 1) developer naming conventions (e.g., URL paths) of ten offer hints as to data collection purposes, and 2) external knowledge, such as app metadata and information about the domain name, are meaningful cues that can be used to infer the behavior of different traffic requests. MobiPurpose parses each traffic request body into key-value pairs, and infers the data type and data collection purpose of each key-value pair using a combination of supervised learning and text pattern bootstrapping. We evaluated MobiPurpose's effectiveness using a dataset cross-labeled by ten human experts. Our results show that MobiPurpose can predict the data collection purpose with an average precision of 84\% (among 19 unique categories).\cite{jin2018they}

\medskip
—It is a new world that we are living in: The “App
Generation” has come. The term “app” is a shortening of the
term “mobile application.” It refers to software applications
designed to run on smartphones, tablet computers and other
mobile devices. Most mobile apps are free, yet the increasing
growth of apps has yielded a number of different revenue
models to reap huge profits, such as the instant messaging app
WhatsApp Messenger and the gaming app Puzzle \& Dragons.
Studies of these app business models (ABMs) have not been
extensive in the literature. Abundant research has examined
apps as a promotional tool in mobile advertising or mobile
marketing, but not as a business model to generate revenue.
The app business is an evolving market and research on ABMs
is essential to reveal the contemporary situation and critical
success factors in implementing and monetizing an ABM. This
study aims to investigate whether there exist different ABMs
and what factors app users consider important when they use
and pay for an app. In-depth interviews and focus groups with
apps users and app enterprises were carried out. We found app
users have different attitudes about and evaluations of various
types of apps. Users look more for utilitarian benefits such as
aesthetic appeal and perceived ease of use in apps such as maps,
news and fitness; while they focus more on hedonic benefits
such as personal emotional attachment and achievement
component in gaming and social media apps. The findings of
this study will provide insight to practitioners in developing
features and benefits to meet app users’ increasing
expectations and requirements. \cite{tang2016mobile}

\medskip
In the current age of the Fourth Industrial Revolution (4IR or Industry 4.0), the digital world has a wealth of data, such as Internet of Things (IoT) data, cybersecurity data, mobile data, business data, social media data, health data, etc. To intelligently analyze these data and develop the corresponding smart and automated applications, the knowledge of artificial intelligence (AI), particularly, machine learning (ML) is the key. Various types of machine learning algorithms such as supervised, unsupervised, semi-supervised, and reinforcement learning exist in the area. Besides, the deep learning, which is part of a broader family of machine learning methods, can intelligently analyze the data on a large scale. In this paper, we present a comprehensive view on these machine learning algorithms that can be applied to enhance the intelligence and the capabilities of an application. Thus, this study’s key contribution is explaining the principles of different machine learning techniques and their applicability in various real-world application domains, such as cybersecurity systems, smart cities, healthcare, e-commerce, agriculture, and many more. We also highlight the challenges and potential research directions based on our study. Overall, this paper aims to serve as a reference point for both academia and industry professionals as well as for decision-makers in various real-world situations and application areas, particularly from the technical point of view. \cite{sarker2021machine}

\medskip

This paper presents the uses and effect of mobile
application in individuals, business and social area. In modern
information and communication age mobile application is one of
the most concerned and rapidly developing areas. This paper
demonstrates that how individual mobile user facilitate using
mobile application and the popularity of the mobile application.
Here we are presenting the consequence of mobile application in
business sector. Different statistical data of the past and present
situation of mobile application from different parts of the world
has been presented here to express the impact. This paper also
presents some effect of mobile application on society from the
ethical perspective. \cite{islam2010mobile}

\newpage

\medskip

Software organizations are nowadays facing increased demand for modernizing their legacy software
systems using up-to-date technologies. The combination of Model-Driven Development and delivery
models like Cloud and Software as a Service have become a very promising approach for software
modernization that possesses a lot of advantages, including great deal of automation and reuse of system
functionality. However, the use of such new and immature technologies is very challenging and requires a
comprehensive methodology for their seamless application within the software modernization projects.
When developing such methodology, questions on whether agile methods and techniques should be
incorporated and what could be the benefits and implications from that become of particular interest. To
help answering these questions, the paper evaluates the potential of agile methods and techniques to address
the challenges of Model-Driven Modernization. The challenges are extracted through a systematic review of
the existing body of literature on Model-Driven Development and Software Modernization, and the
evaluation is conducted through the Delphi technique. As a result, a ranked list of applicable agile
techniques is proposed and suggestions for their use in Model-Driven Modernization are made.\cite{stavru2013challenges}


\medskip

Umple is an open-source software modeling tool and compiler. It incorporates textual language constructs for UML modeling, including associations and state machines. It includes traits, aspects, and mixins for separation of concerns. It supports embedding methods written in many object-oriented languages, enabling it to generate complete multilingual systems. It provides comprehensive analysis of models and generates many kinds of diagrams, some of which can be edited to update the Umple code. Umple runs on the command line, in a web browser or in integrated development environments. It is designed to help developers reduce code volume, while they develop in an agile, model-driven manner. Umple is also targeted at educational users where students are motivated by its ability to generate real systems from their software models. \cite{lethbridge2021umple}

\medskip

In a world where the industry of mobile applications is continuously expanding and new health care apps and devices are created every day, it is important to take special care of the collection and treatment of users’ personal health information. However, the appropriate methods to do this are not usually taken into account by apps designers and insecure applications are released. This paper presents a study of security and privacy in mHealth, focusing on three parts: a study of the existing laws regulating these aspects in the European Union and the United States, a review of the academic literature related to this topic, and a proposal of some recommendations for designers in order to create mobile health applications that satisfy the current security and privacy legislation. This paper will complement other standards and certifications about security and privacy and will suppose a quick guide for apps designers, developers and researchers. \cite{martinez2015privacy}

The mobile application (app) industry has grown tremendously over the past ten years, primarily fueled by small app development businesses. Lacking advertising budgets, these small and relatively unknown businesses often offer free versions of their paid apps to be noticed in the crowded app industry and to reduce customer uncertainty about app quality and fit. The authors build on the existing marketing and information systems literature on sampling and versioning to investigate the implications of offering free versions for the adoption speed of paid apps. Using a unique data set of 7.7 million observations from 12,315 paid apps, and accounting for endogeneity, the authors find that although the practice of offering free versions of paid apps is popular, it is negatively associated with paid app adoption speed. They also find that this negative association between free version presence and paid app adoption speed is stronger both for hedonic apps and in the later life stages of paid apps. The authors hope that the study's results will encourage app developers to reevaluate their current strategy of offering free versions of paid apps and prompt academics to produce more work focusing on this industry.\cite{arora2017implications}

\printbibliography

\end{document}
