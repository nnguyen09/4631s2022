\documentclass{article}
\usepackage[
backend=biber,
style=alphabetic,
sorting=ynt
]{biblatex}
\usepackage{indentfirst}



\addbibresource{bibliography.bib}

\title{Bibliography Summary}
\author{Ngan Nguyen}
\date{January 2022}

\begin{document}


\maketitle

\section*{Summary}
% First summary: Evaluation of mobile app paradigms 
In the research paper Evaluation of mobile app paradigms \cite{10.1145/2428955.2428968}, the authors main purpose is to help the audiences understand the four main mobile application paradigms which is mobile native applications, mobile widgets, mobile web applications, and HTML5 mobile applications. "A mobile native application or native app is an application
specifically developed to execute on a specific device platform" \cite{10.1145/2428955.2428968}. Mobile widget is a quicker way to access new, podcast, or weather content. Mobile web application is used to deliver information and services. The World Wide Web Consortium (W3C) has defined HTML5 as a set of capabilities that can do all of the jobs that existing technologies can handle in mobile Web apps. The authors evaluation: for developer, the easiest application to develop is widget; for user, the easiest application to use is native apps; for the service provider, HTML5 have the advantage of distribution. In conclusion, the authors believe that "HTML5 mobile apps keep their first places in the race of mobile paradigms" \cite{10.1145/2428955.2428968}. The authors also predict that mobile web app will soon become history and replaced by HTML5 mobile app. 

%used to skip line 
\medskip

% Second summary: Software engineering issues for mobile application development
The paper "Software engineering issues for mobile application development" \cite{10.1145/1882362.1882443} presents an overview of key software engineering research challenges relating to the development of mobile app development. The authors pointed out some additional requirement that mobile applications need such as potentially interact wth other apps, sensing handling, native and hybrid, security, complexity of testing, and power consumption. The paper suggests the best practice for mobile application is agile approach. The authors think that there are a lot of space for researching in mobile development area namely user-experience , non-functional requirements, as well as process, tool, and architecture. "While the large number of mobile applications makes it appear
that software development processes for them are well understood, there remain a large number of complex issues where further work is needed."\cite{10.1145/1882362.1882443}, the author concluded.






\printbibliography

\end{document}
