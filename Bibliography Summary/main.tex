\documentclass{article}
\usepackage[
backend=biber,
style=alphabetic,
sorting=ynt
]{biblatex}
\usepackage{indentfirst}



\addbibresource{bibliography.bib}

\title{Bibliography Summary}
\author{Ngan Nguyen}
\date{January 2022}

\begin{document}


\maketitle

\section*{Summary}
% First summary: Evaluation of mobile app paradigms 
In the research paper Evaluation of mobile app paradigms \cite{10.1145/2428955.2428968}, the authors main purpose is to help the audiences understand the four main mobile application paradigms which is mobile native applications, mobile widgets, mobile web applications, and HTML5 mobile applications. "A mobile native application or native app is an application
specifically developed to execute on a specific device platform" \cite{10.1145/2428955.2428968}. Mobile widget is a quicker way to access new, podcast, or weather content. Mobile web application is used to deliver information and services. The World Wide Web Consortium (W3C) has defined HTML5 as a set of capabilities that can do all of the jobs that existing technologies can handle in mobile Web apps. The authors evaluation: for developer, the easiest application to develop is widget; for user, the easiest application to use is native apps; for the service provider, HTML5 have the advantage of distribution. In conclusion, the authors believe that "HTML5 mobile apps keep their first places in the race of mobile paradigms" \cite{10.1145/2428955.2428968}. The authors also predict that mobile web app will soon become history and replaced by HTML5 mobile app. 

%used to skip line 
\medskip

% Second summary: Software engineering issues for mobile application development
The paper "Software engineering issues for mobile application development" \cite{10.1145/1882362.1882443} presents an overview of key software engineering research challenges relating to the development of mobile app development. The authors pointed out some additional requirement that mobile applications need such as potentially interact with other apps, sensing handling, native and hybrid, security, complexity of testing, and power consumption. The paper suggests the best practice for mobile application is agile approach. The authors think that there are a lot of space for researching in mobile development area namely user-experience , non-functional requirements, as well as process, tool, and architecture. "While the large number of mobile applications makes it appear
that software development processes for them are well understood, there remain a large number of complex issues where further work is needed."\cite{10.1145/1882362.1882443}, the author concluded.

%used to skip line 
\medskip

In the paper "What can Android mobile app developers do about the energy consumption of machine learning?", the author introduce some API and library that used to implement machine learning algorithm with mobile app. One example of the framework that can use machine learning algorithm with mobile app is Neural Networks API. Apple has released CoreML (Inc. 2017) for iOS \cite{mcintosh2019can}. The author also introduce apps that employ machine learning like Google Play Crawler and App Brain. One of the main concern of using such an enormous amount of data is the energy that would cost for the mobile device. The author concluded that Naïve Bayes and J48 are the best algorithm implementations tested to use for applications trying to reduce energy use \cite{mcintosh2019can}. If we're focus more on accuracy, the author suggested MLP had the highest average accuracy overall, with an average classification accuracy of over 95\% and an average kappa of over 0.92. \cite{mcintosh2019can}. The researcher concluded that different learning algorithm do different thing, it is up to developers choose how to use it.


%used to skip line 
\medskip

In the article "The mobile apps industry: A case study", the author discussed about the evolution, the basic, the impact of the mobile app industry. The different between Android app and iPhone app \cite{rakestraw2013mobile}. They also discussed the market place of big tech like Facebook, Amazon, Niche and the consumer reference. Revenue that these app generated. Network provider play an important role in order for these big tech to be successful. The author also brought up security and privacy, the primary user concerned. They also did some research about trends and predict the future. Based on the research, mobile app consumption has beaten web application consumption by 8\%. 

\medskip

Many smartphone apps collect potentially sensitive personal data and send it to cloud servers. However, most mobile users have a poor understanding of why their data is being collected. We present MobiPurpose, a novel technique that can take a network request made by an Android app and then classify the data collection purposes, as one step towards making it possible to explain to non-experts the data disclosure contexts. Our purpose inference works by leveraging two observations: 1) developer naming conventions (e.g., URL paths) of ten offer hints as to data collection purposes, and 2) external knowledge, such as app metadata and information about the domain name, are meaningful cues that can be used to infer the behavior of different traffic requests. MobiPurpose parses each traffic request body into key-value pairs, and infers the data type and data collection purpose of each key-value pair using a combination of supervised learning and text pattern bootstrapping. We evaluated MobiPurpose's effectiveness using a dataset cross-labeled by ten human experts. Our results show that MobiPurpose can predict the data collection purpose with an average precision of 84\% (among 19 unique categories).\cite{jin2018they}

\medskip
—It is a new world that we are living in: The “App
Generation” has come. The term “app” is a shortening of the
term “mobile application.” It refers to software applications
designed to run on smartphones, tablet computers and other
mobile devices. Most mobile apps are free, yet the increasing
growth of apps has yielded a number of different revenue
models to reap huge profits, such as the instant messaging app
WhatsApp Messenger and the gaming app Puzzle \& Dragons.
Studies of these app business models (ABMs) have not been
extensive in the literature. Abundant research has examined
apps as a promotional tool in mobile advertising or mobile
marketing, but not as a business model to generate revenue.
The app business is an evolving market and research on ABMs
is essential to reveal the contemporary situation and critical
success factors in implementing and monetizing an ABM. This
study aims to investigate whether there exist different ABMs
and what factors app users consider important when they use
and pay for an app. In-depth interviews and focus groups with
apps users and app enterprises were carried out. We found app
users have different attitudes about and evaluations of various
types of apps. Users look more for utilitarian benefits such as
aesthetic appeal and perceived ease of use in apps such as maps,
news and fitness; while they focus more on hedonic benefits
such as personal emotional attachment and achievement
component in gaming and social media apps. The findings of
this study will provide insight to practitioners in developing
features and benefits to meet app users’ increasing
expectations and requirements. \cite{tang2016mobile}

\medskip
In the current age of the Fourth Industrial Revolution (4IR or Industry 4.0), the digital world has a wealth of data, such as Internet of Things (IoT) data, cybersecurity data, mobile data, business data, social media data, health data, etc. To intelligently analyze these data and develop the corresponding smart and automated applications, the knowledge of artificial intelligence (AI), particularly, machine learning (ML) is the key. Various types of machine learning algorithms such as supervised, unsupervised, semi-supervised, and reinforcement learning exist in the area. Besides, the deep learning, which is part of a broader family of machine learning methods, can intelligently analyze the data on a large scale. In this paper, we present a comprehensive view on these machine learning algorithms that can be applied to enhance the intelligence and the capabilities of an application. Thus, this study’s key contribution is explaining the principles of different machine learning techniques and their applicability in various real-world application domains, such as cybersecurity systems, smart cities, healthcare, e-commerce, agriculture, and many more. We also highlight the challenges and potential research directions based on our study. Overall, this paper aims to serve as a reference point for both academia and industry professionals as well as for decision-makers in various real-world situations and application areas, particularly from the technical point of view. \cite{sarker2021machine}

\medskip

This paper presents the uses and effect of mobile
application in individuals, business and social area. In modern
information and communication age mobile application is one of
the most concerned and rapidly developing areas. This paper
demonstrates that how individual mobile user facilitate using
mobile application and the popularity of the mobile application.
Here we are presenting the consequence of mobile application in
business sector. Different statistical data of the past and present
situation of mobile application from different parts of the world
has been presented here to express the impact. This paper also
presents some effect of mobile application on society from the
ethical perspective. \cite{islam2010mobile}

\newpage

\medskip

Software organizations are nowadays facing increased demand for modernizing their legacy software
systems using up-to-date technologies. The combination of Model-Driven Development and delivery
models like Cloud and Software as a Service have become a very promising approach for software
modernization that possesses a lot of advantages, including great deal of automation and reuse of system
functionality. However, the use of such new and immature technologies is very challenging and requires a
comprehensive methodology for their seamless application within the software modernization projects.
When developing such methodology, questions on whether agile methods and techniques should be
incorporated and what could be the benefits and implications from that become of particular interest. To
help answering these questions, the paper evaluates the potential of agile methods and techniques to address
the challenges of Model-Driven Modernization. The challenges are extracted through a systematic review of
the existing body of literature on Model-Driven Development and Software Modernization, and the
evaluation is conducted through the Delphi technique. As a result, a ranked list of applicable agile
techniques is proposed and suggestions for their use in Model-Driven Modernization are made.\cite{stavru2013challenges}


\medskip

Umple is an open-source software modeling tool and compiler. It incorporates textual language constructs for UML modeling, including associations and state machines. It includes traits, aspects, and mixins for separation of concerns. It supports embedding methods written in many object-oriented languages, enabling it to generate complete multilingual systems. It provides comprehensive analysis of models and generates many kinds of diagrams, some of which can be edited to update the Umple code. Umple runs on the command line, in a web browser or in integrated development environments. It is designed to help developers reduce code volume, while they develop in an agile, model-driven manner. Umple is also targeted at educational users where students are motivated by its ability to generate real systems from their software models. \cite{lethbridge2021umple}

\medskip

In a world where the industry of mobile applications is continuously expanding and new health care apps and devices are created every day, it is important to take special care of the collection and treatment of users’ personal health information. However, the appropriate methods to do this are not usually taken into account by apps designers and insecure applications are released. This paper presents a study of security and privacy in mHealth, focusing on three parts: a study of the existing laws regulating these aspects in the European Union and the United States, a review of the academic literature related to this topic, and a proposal of some recommendations for designers in order to create mobile health applications that satisfy the current security and privacy legislation. This paper will complement other standards and certifications about security and privacy and will suppose a quick guide for apps designers, developers and researchers. \cite{martinez2015privacy}

The mobile application (app) industry has grown tremendously over the past ten years, primarily fueled by small app development businesses. Lacking advertising budgets, these small and relatively unknown businesses often offer free versions of their paid apps to be noticed in the crowded app industry and to reduce customer uncertainty about app quality and fit. The authors build on the existing marketing and information systems literature on sampling and versioning to investigate the implications of offering free versions for the adoption speed of paid apps. Using a unique data set of 7.7 million observations from 12,315 paid apps, and accounting for endogeneity, the authors find that although the practice of offering free versions of paid apps is popular, it is negatively associated with paid app adoption speed. They also find that this negative association between free version presence and paid app adoption speed is stronger both for hedonic apps and in the later life stages of paid apps. The authors hope that the study's results will encourage app developers to reevaluate their current strategy of offering free versions of paid apps and prompt academics to produce more work focusing on this industry.\cite{arora2017implications}

\medskip

Explorations	into	today’s	labour	context	reveal	a	wide	schism	between	those	workers	who	live	under	
conditions	of	precarity and	contingency	and	those	who	seem	to	be	living	the	dream	– and	not	only	in	
terms	of	wages.	The	standardized	work	day	and	Taylorized	division	of	labour	that	characterized	most	
of	the	industrial	era	has	transitioned,	at	least	in	large	part,	into	a	regime	of	flexibility	and	insecurity	
that	reconstitutes	not	only	working	but	lifestyle	conditions.	This	paper	is	intended	as	an	initial	conceptual	investigation	of	a	dual	trend	in	the	conditions	of	labour	under	digital capitalism:	the	rise	of	
contractual	contingency	and	insecurity	and	the	introduction	of	fun	and	hipness	into	the	office	environment as	a	means	of	work	intensification. \cite{anderson2016contingency}

\medskip

This paper provides a multi-disciplinary overview of the research issues and achievements in the field of Big Data and its visualization techniques and tools. The main aim is to summarize challenges in visualization methods for existing Big Data, as well as to offer novel solutions for issues related to the current state of Big Data Visualization. This paper provides a classification of existing data types, analytical methods, visualization techniques and tools, with a particular emphasis placed on surveying the evolution of visualization methodology over the past years. Based on the results, we reveal disadvantages of existing visualization methods. Despite the technological development of the modern world, human involvement (interaction), judgment and logical thinking are necessary while working with Big Data. Therefore, the role of human perceptional limitations involving large amounts of information is evaluated. Based on the results, a non-traditional approach is proposed: we discuss how the capabilities of Augmented Reality and Virtual Reality could be applied to the field of Big Data Visualization. We discuss the promising utility of Mixed Reality technology integration with applications in Big Data Visualization. Placing the most essential data in the central area of the human visual field in Mixed Reality would allow one to obtain the presented information in a short period of time without significant data losses due to human perceptual issues. Furthermore, we discuss the impacts of new technologies, such as Virtual Reality displays and Augmented Reality helmets on the Big Data visualization as well as to the classification of the main challenges of integrating the technology.

\medskip

Mobile apps are increasingly realized by using a cross-platform development framework. Using such frameworks, code is written once but the app can be deployed to multiple platforms. Despite progress in research on cross-platform techniques, results (i.e. apps) are not always satisfactory. They are subject to tedious tailoring and the development effort tends to be notable. In these cases, either pure web apps (realized through web browsers) or native apps (realized for each platform separately) are chosen. Recent activities have led to new approaches. In this paper, we have a closer look at three of these, namely React Native, the Ionic Framework, and Fuse. We present a comprehensive analysis of the three approaches. Our work is based on a real-world use case, which allows us to provide generalizable advice. Our findings suggest that there is no clear winner; the frameworks incorporate notable ideas and general progress in the field can be asserted. \cite{majchrzak2017comprehensive}

\medskip

Mobile apps (applications) provide services such as information dissemination, knowledge promotion,
social media integration, and online shopping, all of which are platforms that enable the strengthening
of communication and foster interaction between firms and consumers. The related pieces of literature
on the continuous use of mobile apps have noted that apps must have convenience, unique value, social
value, incentives, entertainment, and other such qualities to attract continuous usage by customers. As
such, stickiness has become the key factor in the business success of apps. For businesses managing
mobile apps, the topics of how to design content for mobile device users, how to measure media value
and returns, and how to capture the attention of users and make them willing to spend more time using
the application are all worthy of exploration. This study proposes the analytical model for mobile app
stickiness to measure the significance of various influencing factors of the hierarchical structure and to
measure the performance of mobile apps. The study explored key factors that affected user stickiness
while examining usage statistics to develop management strategies geared toward mobile app sticki-
ness to improve customer/user loyalty. The proposed model can function as a tool for app planners in
measuring user stickiness and app performance while serving as reference for future studies into app
stickiness and real-life applications. It can also clarify the influence of key factors influencing app stick-
iness, which can help app planners develop appropriate strategies and function as a reference point for
future improvement and optimization strategies. \cite{hsu2020development}

\medskip

Mobile cross-platform tools (CPTs) provide an interesting
alternative to native development. Cross-platform tools aim
at sharing a significant portion of the application codebase
between the implementations for the different platforms.
This can drastically decrease the development costs of mobile applications. There is, however, some reluctance of mobile application developers to adopt these tools. One of the
reasons is that the landscape of CPTs is so diverse that it is
hard to select the most suitable CPT to implement a specific
application. The contribution of this paper is twofold. First,
it presents a performance analysis of a fully functional mobile application implemented with ten cross-platform tools
and native for Android, iOS and Windows Phone. The performance tests are executed on a high- and low-end Android
and iOS device, and a Windows Phone device. Second,
based on the performance analysis, general conclusions of
which application developers should be aware when selecting a specific (type of) cross-platform tool are drawn. \cite{willocx2016comparing}

\medskip

Developing energy efficient mobile applications is an important goal for software developers as energy usage can directly affect the usability of a mobile device. Unfortunately,
developers lack guidance as to how to improve the energy
efficiency of their implementation and which practices are
most useful. In this paper we conducted a small-scale empirical evaluation of commonly suggested energy-saving and
performance-enhancing coding practices. In the evaluation
we evaluated the degree to which these practices were able
to save energy as compared to their unoptimized code counterparts. Our results provide useful guidance for mobile app
developers. In particular, we found that bundling network
packets up to a certain size and using certain coding practices for reading array length information, accessing class
fields, and performing invocations all led to reduced energy
consumption. However, other practices, such as limiting
memory usage had a very minimal impact on energy usage. These results serve to inform the developer community
about specific coding practices that can help lower the overall energy consumption and improve the usability of their
applications. \cite{li2014investigation}

\medskip

In this paper, The mobile application field has been receiving astronomical attention from the
past few years due to the growing number of mobile app downloads and withal due to the
revenues being engendered .With the surge in the number of apps, the number of lamentable
apps/failing apps has withal been growing.Interesting mobile app statistics are included in this
paper which might avail the developers understand the concerns and merits of mobile apps.The
authors have made an effort to integrate all the crucial factors that cause apps to fail which
include negligence by the developers, technical issues, inadequate marketing efforts, and high
prospects of the users/consumers.The paper provides suggestions to eschew failure of apps. As
per the various surveys, the number of lamentable/failing apps is growing enormously, primarily
because mobile app developers are not adopting a standard development life cycle for the
development of apps. In this paper, we have developed a mobile application with the aid of
traditional software development life cycle phases (Requirements, Design, Develop, Test, and,
Maintenance) and we have used UML, M-UML, and mobile application development
technologies. \cite{inukollu2014factors}

An effective development model can help improve competitive advantage and shorten release cycles,
which is vital in the fast paced environment of mobile app development.
Objective: The aim with this paper is to provide an extensive review of existing mobile app development
models.
Method: The review is done by following a systematic literature review process. Also presented is an assessment of the usefulness and relevance to industry of the models based on a rigor and relevance framework.
Results: 20 primary studies were identified, each with distinct models. Agile methods or state-based principles
are commonly adopted across the models. Relatively little effort focuses on deployment, maintenance, project
evaluation activities.
Conclusion: The review reveals that the contexts in which the identified models are intended to be used vary.
This benefits practitioners as they are able to select a model that suits their contexts. However, the usefulness in
industry of most of the models, based on the contexts in which the models were evaluated, is questionable. There
is a need for evaluating mobile app models in contexts that resemble realistic contexts. The review also calls for
further research addressing special constraints of mobile apps, e.g., testing apps on multiple-platforms, user
involvement in release planning and continuous deployment.\cite{jabangwe2018software}

\medskip
Advent of computationally efficient smartphones, inexpensive high-resolution cameras, drones, and robotic sensors has brought a new era of next-generation intelligent monitoring systems for civil infrastructure. Vibration-based condition assessment has garnered as a prominent method of evaluating the health of large-scale infrastructure. The use of contact-based sensors for acquiring vibration data becomes uneconomical and tedious due to their instrumentation cost, centralized nature, and densification required to collect sufficient data for system identification of modern complex structures. A need to advance and develop alternative methods for efficient sensing system results in next-generation measurement technology of structural health monitoring. The abundance of handheld smartphones with easily programmable framework has helped in modifying relevant software to acquire vibration data using embedded sensors in the smartphone. The inexpensive cameras have been used to capture images and videos that are utilized to understand the structural behavior with the aid of advanced signal processing techniques. The inaccessible components of structures require noncontact sensors such as unmanned aerial vehicles (UAVs) or so-called drones and mobile sensors to acquire structural data. To the authors' knowledge, this paper first time presents a comprehensive review of a suite of next-generation smart sensing technology that has been developed in recent years within the context of structural health monitoring. The state-of-the-art methods have been presented by conducting a detailed literature review of the recent applications of smartphones, UAVs, cameras, and robotic sensors used in acquiring and analyzing the vibration data for structural condition monitoring and maintenance.

\medskip
Information about vehicle safety, such as the driving safety status and the road safety index, is of great importance to protect humans and support safe driving route planning. Despite some research on driving safety analysis, the accuracy and granularity of driving safety assessment are both very limited. Also, the problem of precisely and dynamically predicting road safety throughout a city has not been sufficiently studied and remains open. With the proliferation of sensor-equipped vehicles and smart devices, a huge amount of mobile sensing data provides an opportunity to conduct vehicle safety analysis. In this article, we first discuss mobile sensing data collection in VANETs and then identify two main challenges in vehicle safety analysis in VANETs, i.e., driving safety analysis and road safety analysis. In each issue, we review and classify the state-of-the-art vehicle safety analysis techniques into different categories. For each category, a short description is given followed by a discussion of limitations. In order to improve vehicle safety, we propose a new deep learning framework (DeepRSI) to conduct real-time road safety prediction from the data mining perspective. Specifically, the proposed framework considers the spatio-temporal relationship of vehicle GPS trajectories and external environment factors. The evaluation results demonstrate the advantages of our proposed scheme over other methods by utilizing mobile sensing data collected in VANETs.\cite{peng2018vehicle}

\medskip

Cameras are becoming ubiquitous in the Internet of Things (IoT) and can use face recognition technology to improve context. There is a large accuracy gap between today’s publicly available face
recognition systems and the state-of-the-art private face recognition systems. This paper presents
our OpenFace face recognition library that bridges this accuracy gap. We show that OpenFace provides near-human accuracy on the LFW benchmark and present a new classification benchmark
for mobile scenarios. This paper is intended for non-experts interested in using OpenFace and
provides a light introduction to the deep neural network techniques we use\cite{amos2016openface}

\medskip

In this paper, we investigate various algorithms for
face recognition on mobile phones. First step in any face
recognition system is face detection. We investigated
algorithms like color segmentation, template matching etc. for
face detection, and Eigen & Fisher face for face recognition.
The algorithms have been first profiled in MATLAB and then
implemented on the DROID phone. While implementing the
algorithms, we made a tradeoff between accuracy and
computational complexity of the algorithm mainly because we
are implementing the face recognition system on a mobile
phone with limited hardware capabilities. \cite{dave2010face}
\printbibliography

\end{document}
